% Copyright 2004 by Till Tantau <tantau@users.sourceforge.net>.
%
% In principle, this file can be redistributed and/or modified under
% the terms of the GNU Public License, version 2.
%
% However, this file is supposed to be a template to be modified
% for your own needs. For this reason, if you use this file as a
% template and not specifically distribute it as part of a another
% package/program, I grant the extra permission to ntely copy and
% modify this file as you see fit and even to delete this copyright
% notice. 

\documentclass{beamer}

% There are many different themes available for Beamer. A comprehensive
% list with examples is given here:
% http://deic.uab.es/~iblanes/beamer_gallery/index_by_theme.html
% You can uncomment the themes below if you would like to use a different
% one:
%\usetheme{AnnArbor}
%\usetheme{Antibes}
%\usetheme{Bergen}
%\usetheme{Berkeley}
%\usetheme{Berlin}
\usetheme{Boadilla}
%\usetheme{boxes}
%\usetheme{CambridgeUS}
%\usetheme{Copenhagen}
%\usetheme{Darmstadt}
%\usetheme{default}
%\usetheme{Frankfurt}
%\usetheme{Goettingen}
%\usetheme{Hannover}
%\usetheme{Ilmenau}
%\usetheme{JuanLesPins}
%\usetheme{Luebeck}
%\usetheme{Madrid} % the best one
%\usetheme{Malmoe}
%\usetheme{Marburg}
%\usetheme{Montpellier}
%\usetheme{PaloAlto}
%\usetheme{Pittsburgh} %no up-panel good 
%\usetheme{Rochester}
%\usetheme{Singapore}
%\usetheme{Szeged}
%\usetheme{Warsaw}

\usepackage{amsmath}
\usepackage{amssymb}
\usepackage{subfigure}
%\usepackage{algorithm}
%\usepackage{algorithmic}
\usepackage{tabu}

\def\A{{\bf A}}
\def\B{{\bf B}}
\def\X{{\bf X}}
\def\C{{\bf C}}
\def\D{{\bf D}}
\def\a{{\bf a}}
\def\b{{\bf b}}
\def\MO{{\mathcal O}}
\def\BR{{\mathbb R}}

\title[FD-AMM]{Frequent Direction Algorithms for Approximate Matrix Multiplication with Applications in CCA}

% A subtitle is optional and this may be deleted
%\subtitle{Optional Subtitle}

\author{Qiaomin Ye \and Luo Luo \and Zhihua Zhang}
% - Give the names in the same order as the appear in the paper.
% - Use the \inst{?} command only if the authors have different
%   affiliation.

\institute[] % (optional, but mostly needed)
{
  Department of Computer Science and Engineering \\
  Shanghai Jiao Tong University
  }
% - Use the \inst command only if there are several affiliations.
% - Keep it simple, no one is interested in your street address.

\date{IJCAI, 2016}
% - Either use conference name or its abbreviation.
% - Not really informative to the audience, more for people (including
%   yourself) who are reading the slides online

\subject{Theoretical Computer Science}
% This is only inserted into the PDF information catalog. Can be left
% out. 

% If you have a file called "university-logo-filename.xxx", where xxx
% is a graphic format that can be processed by latex or pdflatex,
% resp., then you can add a logo as follows:

% \pgfdeclareimage[height=0.5cm]{university-logo}{university-logo-filename}
% \logo{\pgfuseimage{university-logo}}

% Delete this, if you do not want the table of contents to pop up at
% the beginning of each subsection:
\AtBeginSubsection[]
{
  \begin{frame}<beamer>{Outline}
    \tableofcontents[currentsection,currentsubsection]
  \end{frame}
}

% Let's get started
\begin{document}

\begin{frame}
  \titlepage
\end{frame}

\begin{frame}{Outline}
  \tableofcontents
  % You might wish to add the option [pausesections]
\end{frame}

% Section and subsections will appear in the presentation overview
% and table of contents.
\section{Introduction}

%\subsection{First Subsection}


\begin{frame}{Introduction}
	  %  \begin{block}{Approximate matrix multiplication}
	   % 	\includegraphics*[scale=0.278]{figures/test.pdf}  
	   % \end{block}
	          
	  
		\begin{block}{Approximate matrix multiplication}
			we are given $A$, $B$ each with a large number of rows $n$, and the goal is to compute some matrix $X$, s.t. 
			$$\Vert A^TB - X \Vert_X$$ %\centerdot
			is small, for some matrix norm $\Vert \cdot  \Vert_X$. 
		\end{block}
		%	\includegraphics*[scale=0.278]{figures/test.pdf}  
		\pause
		
  \begin{itemize}
  \item {
    Image processing
  }
  \item {
     Information retrieval
  }
  \item{
  	Approximate leverage scores
  	 }
  	 \item { 
  	 	Large-scale $k$-means clustering
  	 	}
  \end{itemize}
  
\end{frame}

\begin{frame}{Introduction}
	  \begin{block}{Approximate matrix multiplication}
	 	\includegraphics*[scale=0.278]{figures/test.pdf}  
	 \end{block}
	
	
%	\begin{block}{Approximate matrix multiplication}
	%	we are given $A$, $B$ each with a large number of rows $n$, and the goal is to compute some matrix $X$, s.t. 
	%	$$\Vert A^TB - X \Vert_X$$ %\centerdot
%		is small, for some matrix norm $\Vert \cdot  \Vert_X$. 
%	\end{block}
	%	\includegraphics*[scale=0.278]{figures/test.pdf}  
	
	\begin{itemize}
		\item {
			Image processing
		}
		\item {
			Information retrieval
		}
		\item{
			Approximate leverage scores
		}
		\item { 
			Large-scale $k$-means clustering
		}
	\end{itemize}
	
\end{frame}

\begin{frame} {Streaming Matrices}
	
	%	\includegraphics*[scale=0.278]{figures/test.pdf}  
	\begin{itemize}
		\item Matrix too large to be stored.
	
		\item Streaming setting:
		\begin{itemize}
			\item Use a small amount of memory.
			\item Make a single pass over the data. 
		\end{itemize}
	\end{itemize}
	\pause 
	$$A^TB=\sum_i^n a_i^Tb_i$$
	\begin{block}{Naive solution}
		Compute $A^TB$ in time $\MO(nm_1m_2)$.% and space $\MO(m_1m_2)$. 
	\end{block}
\end{frame} 

\section{Related Work}
\subsection{Random Selection}
\begin{frame}{Random Selection}
  \begin{itemize}
  \item {
    Random Selection. 
    It randomly picks $\ell$ rows of $A$ and $B$ with probability 
    $$p_i=\Vert a_i \Vert \Vert b_i \Vert /  \sum_{i=1}^n \Vert a_i \Vert \Vert b_i \Vert,$$
    and with high probability, 
    $$\Vert A^TB - C^TD \Vert_2 \leq \frac{\MO(1)}{\sqrt{\ell}} \Vert A \Vert_F \Vert B \Vert_F.$$    
  }
 	\begin{block}{Random Selection}
 		Get $C$ and $D$ in time $\MO(n(m_1+m_2)))$. %and space $\MO(\ell(m_1+m_2))$. 
 	\end{block}
 
  \end{itemize}
\end{frame}

\subsection{Random Projection}
\begin{frame}{Random Projection}
\begin{itemize}
	\item{
		Random Projection. 
	Let $\Pi\in \BR^{\ell \times n}$  be some subspace embedding, then  
	$$\Vert A^TB -  (\Pi A)^T(\Pi B)  \Vert_2 \leq \sqrt{\frac{\MO(r)}{\ell}} \Vert A \Vert_2 \Vert B \Vert_2 $$
	holds with high probability, where $r=rank(A)+rank(B)$. For fast or sparse subspace embedding, $\Pi A$ and $\Pi B$ can be computed quickly. 
}
	\begin{block}{Random Projection}
		For dense $\Pi$, get $C$ and $D$ in time $\MO(n\ell(m_1+m_2))$.  \\% and space $\MO(m_1m_2)$. \\
		For sparse $\Pi$, get $C$ and $D$ in time $\MO(nnz(A)+nnz(B))$. %and space todo
	\end{block}

\end{itemize}
\end{frame}

\section{Methodology}

\subsection{Frequent Directions} 

\begin{frame}{Frequent Directions}
\begin{block}{FD (Edo Liberty 2013)}
	Give $A\in \BR^{d\times n}$, FD efficiently maintains a matrix $B$ with only $\ell$ columns s.t. 
 $$\Vert AA^T  - BB^T \Vert_2 \le 2 \Vert A\Vert_F^2/\ell, $$
\end{block}

The FD algorithm runs in time $\MO(nd\ell)$. %  and space $\MO(d\ell)$.
\end{frame}

\begin{frame}
	\begin{center}
		\begin{tabular}{ccc}
			\includegraphics*[scale=0.23]{figures/FD1.pdf} &
			   \includegraphics*[scale=0.23]{figures/FD2_white.png} 
			&  \includegraphics*[scale=0.23]{figures/FD3_white.png}  \\
			%	(a)  & (b) & (c)   \\
			\includegraphics*[scale=0.23]{figures/FD4_white.png}  
			& \includegraphics*[scale=0.23]{figures/FD5_white.png}  
			 & \includegraphics*[scale=0.23]{figures/FD3_white.png} 
			\\
		\end{tabular} 
	\end{center}
	
	We keep a sketch of $\ell$ columns. 
	
\end{frame}

\begin{frame}
	\begin{center}
		\begin{tabular}{ccc}
			\includegraphics*[scale=0.23]{figures/FD1.pdf} &
			\includegraphics*[scale=0.23]{figures/FD2.pdf} 
			&  \includegraphics*[scale=0.23]{figures/FD3_white.png}  \\
			%	(a)  & (b) & (c)   \\
			\includegraphics*[scale=0.23]{figures/FD4_white.png}  
			& \includegraphics*[scale=0.23]{figures/FD5_white.png}  
			& \includegraphics*[scale=0.23]{figures/FD3_white.png} 
			\\
		\end{tabular} 
	\end{center}
	Input vectors are simply stored in empty columns. 
	
\end{frame}

\begin{frame}
	\begin{center}
		\begin{tabular}{ccc}
			\includegraphics*[scale=0.23]{figures/FD1.pdf} &
			\includegraphics*[scale=0.23]{figures/FD2.pdf} 
			&  \includegraphics*[scale=0.23]{figures/FD3.png}  \\
			%	(a)  & (b) & (c)   \\
			\includegraphics*[scale=0.23]{figures/FD4_white.png}  
			& \includegraphics*[scale=0.23]{figures/FD5_white.png}  
			& \includegraphics*[scale=0.23]{figures/FD3_white.png} 
			\\
		\end{tabular} 
	\end{center}
	When the sketch is 'full' we need to zero out some columns.
	
\end{frame}

\begin{frame}
	\begin{center}
		\begin{tabular}{ccc}
			\includegraphics*[scale=0.23]{figures/FD1.pdf} &
			\includegraphics*[scale=0.23]{figures/FD2.pdf} 
			&  \includegraphics*[scale=0.23]{figures/FD3.png}  \\
			%	(a)  & (b) & (c)   \\
			\includegraphics*[scale=0.23]{figures/FD4.png}  
			& \includegraphics*[scale=0.23]{figures/FD5_white.png}  
			& \includegraphics*[scale=0.23]{figures/FD3_white.png} 
			\\
		\end{tabular} 
	\end{center}
	Using the SVD we compute $B=USV^T$ and set $B_{new}=US$. 
\end{frame}

\begin{frame}
	\begin{center}
		\begin{tabular}{ccc}
			\includegraphics*[scale=0.23]{figures/FD1.pdf} &
			\includegraphics*[scale=0.23]{figures/FD2.pdf} 
			&  \includegraphics*[scale=0.23]{figures/FD3.png}  \\
			%	(a)  & (b) & (c)   \\
			\includegraphics*[scale=0.23]{figures/FD4.png}  
			& \includegraphics*[scale=0.23]{figures/FD5.png}  
			& \includegraphics*[scale=0.23]{figures/FD3_white.png} 
			\\
		\end{tabular} 
	\end{center}
	Let $\sigma = \Vert B_{\ell/ 2} \Vert^2$.
\end{frame}

\begin{frame}
	     \begin{center}
	     	\begin{tabular}{ccc}
	     		\includegraphics*[scale=0.23]{figures/FD1.pdf} &
	     		\includegraphics*[scale=0.23]{figures/FD2.pdf}  &
	     		\includegraphics*[scale=0.23]{figures/FD3.png} \\
	     	%	(a)  & (b) & (c)   \\
	     		\includegraphics*[scale=0.23]{figures/FD4.png} &
	     		\includegraphics*[scale=0.23]{figures/FD5.png}  &
	     		\includegraphics*[scale=0.23]{figures/FD1.pdf} \\
	     		
	     	\end{tabular} 
	     \end{center}
\end{frame}

\subsection{FD-AMM}
\begin{frame}{FD-AMM}
	
	\vspace{0.3cm}
	\renewcommand{\arraystretch}{1.1}
	%\begin{tabu} {|[2pt]l|[2pt]}
	\begin{tabu} to 1.0\textwidth {  X[ll]   }
		\tabucline[1pt]{-}
		\textbf{Algorithm 2} FD-AMM \\
		\tabucline[1pt]{-} 
		\textbf{Input:} $\ell$, $A \in R^{n\times m_1}$, $B\in \BR^{n\times m_2}$ \\ 
		~~~~~\alert{$G = [A\ B]$} \\
		% ~~~~~   \\ 
		~~~~~\alert{$H =$ FD$(\ell, G)$}\\
		~~~~~$C = H_{[\ell],1:m_1}$ \\ 
		~~~~~$D =H_{[\ell],m_1+1:m_1+m_2}$  \\
		~~~~~\textbf{return} $C$, $D$ \\
		\tabucline[1pt]{-}
	\end{tabu}
	\pause
\begin{theorem}
	  If $X$ is the result of applying the FD-AMM algorithm to matrices $A$, $B$, and sketch size $\ell$, then  
	  $$\Vert A^TB-C^TD\Vert_2 \le (\Vert A\Vert_F^2+\Vert B\Vert_F^2)/\alert{\ell}. $$  
	
\end{theorem}	
	
\end{frame}

\section{Experiments}
\begin{frame}{Experiments}
		$\Vert A^TB- C^TD \Vert_2$ as a function of the sketch size $\ell$. 
		\begin{center}
				\includegraphics*[scale=0.8]{figures/ms2.pdf} 
		\end{center}
	
\end{frame}


\begin{frame}
	Running time in second as a function of $\ell$. 
		\begin{center}
			\includegraphics*[scale=0.8]{figures/time.pdf} 
		\end{center}
\end{frame}

\begin{frame} 
	\begin{center}
		\large {Thank You}
	\end{center}
	
\end{frame}

\end{document}


